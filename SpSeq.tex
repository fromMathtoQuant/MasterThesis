\documentclass[../main.tex]{subfiles}

\title{Appendice - Spectral Sequences}
\author{Edoardo Manini}
\date{Febbraio 26, 2025}

\begin{document}
\ifSubfilesClassLoaded{
\maketitle
\tableofcontents
}{}

Let $R$ be a ring.

\begin{defn}[of Spectral Sequence] \label{SpSeqDef1}
A spectral sequence consists of a sequence  $\{E_r,d_r\}_{r\geq 2}$ of differential $R$-modules, i.e. modules with differentials $d_r\colon E_r\to E_r$ (morphisms s.t. $d_r^2 = 0$ )  such that $E_{r+1}$ is the homology of $(E_r,d_r)$ i.e.  $E_{r+1} = \frac{Ker(d_r)}{Im(d_r)}$.
\end{defn}
If the sequence stabilizes to $E_{\infty}$, i.e. $\exists l \in \NN $ s.t. $E_{l} = E_{l+1} = \cdots = E_{\infty}$ and $E_{\infty} = GH = \bigoplus_{p}\frac{F^pH}{F^{p+1}H}$ is the associated graded of some filtered module $H$, then we say that the spectral sequence converges to $H$ and we write $E_2\Rightarrow H$.





\begin{defn}[of first quadrant Cohomological type Spectral Sequence]
\label{ss4:cohotype}

A cohomological type spectral sequence consists of a sequence of bigraded differential modules  $\{E_r,d_r\}_{r\geq 2}$ such that $E_{r+1}$ is the cohomology of $(E_r,d_r)$ and such that $d_r$ has bidegree $(r,1-r)$, $d_r\colon E^r_{p,q}\to E^r_{p+r,q+1-r}$, as in these pictures:

\begin{minipage}{0.3\textwidth}
\begin{align*}
&\xymatrix@=0pt{
 \bullet & \bullet & \bullet & \bullet & \bullet \\
 \bullet & \bullet & \bullet & \bullet & \bullet \\
 \bullet & \bullet & \ar@{<-}[]+C;[llu]+C\bullet & \bullet & \ar@{<-}[]+C;[llu]+C\bullet \\
 \bullet & \bullet & \bullet & \bullet & \bullet \\
 \bullet & \bullet & \bullet & \ar@{<-}[]+C;[llu]+C\bullet & \bullet \\
}\\
&\hspace{10pt}(E_2,d_2)
\end{align*}
\end{minipage}
\begin{minipage}{0.3\textwidth}
\begin{align*}
&\xymatrix@=0pt{
 \bullet & \bullet & \bullet & \bullet & \bullet \\
 \bullet & \bullet & \bullet & \bullet & \bullet \\
 \bullet & \bullet & \bullet & \bullet & \ar@{<-}[]+C;[llluu]+C\bullet \\
 \bullet & \bullet & \bullet & \ar@{<-}[]+C;[llluu]+C\bullet & \bullet \\
 \bullet & \bullet & \bullet & \bullet & \bullet \\
}\\
&\hspace{10pt}(E_3,d_3)
\end{align*}
\end{minipage}
\begin{minipage}{0.3\textwidth}
\begin{align*}
&\xymatrix@=0pt{
 \bullet & \bullet & \bullet & \bullet & \bullet \\
 \bullet & \bullet & \bullet & \bullet & \bullet \\
 \bullet & \bullet & \bullet & \bullet & \bullet \\
 \bullet & \bullet & \bullet & \bullet &  \ar@{<-}[]+C;[lllluuu]+C\bullet \\
 \bullet & \bullet & \bullet & \bullet & \ar@{<-}[]+C;[lllluuu]+C\bullet \\
}\\
&\hspace{10pt}(E_4,d_4)
\end{align*}
\end{minipage}
The spectral sequence is first quadrant if $E^r_{p,q} = 0$ for $p,q <0 $.
\end{defn}

In this situation, we say that the Spectral Sequence converges to $H=H^*$ a graded $R$-module if $E_{\infty}^{p,q} \cong \frac{F^pH^{p+q}}{F^{p+1}H^{p+q}}$ and we write
 $E_2^{p,q}\Rightarrow H^{p+q}$. Convergence means that the pages eventually stabilize at each particular position ($p,q$) and that  you can recover the $R$-module $H^n$ from the ``anti-diagonal'' $R$-modules $\{E_\infty^{p,q}\}_{p+q=n}$ via a finite number of extensions. More precisely, there exists a finite \emph{decreasing} filtration of $H^n$ by $R$-modules:
\[
0=A^{r+1}\subseteq A^r\subseteq A^{r-1}\subseteq\ldots\subseteq A^{1}\subseteq A^0=H^n,
\]
together with short exact sequences of $R$-modules:
\begin{align}
&A^{1}\to H^n \to E_\infty^{0,n}\label{equ:filtrationcohoonetotaldimension}\\
&A^{2}\to A^{1} \to E_\infty^{1,n-1}\nonumber\\
&\ldots\nonumber \\
&A^r\to A^{r-1}\to E_\infty^{r-1,n-r+1}\nonumber \\
&0\to A^r\to E_\infty^{r,n-r}.\nonumber
\end{align}






\subsection{Spectral Sequences via exact couples}


\begin{defn}[of Exact Couple]
\label{def:exact-couple}
  An \emph{exact couple} $\calC$ consists of a diagram in $Ab$ exact at each vertex:  
\[
\xymatrix{
&E \ar[rd]^k& \\
D\ar[ur]^j && D\ar[ll]^i 
}
\]
so that $\ker i= \ensuremath{\operatorname{Im}} k$ , $\ker j = \ensuremath{\operatorname{Im}} i$ , $\ker k = \ensuremath{\operatorname{Im}} j$.
 
\end{defn}
Given an exact couple $\calC$:
\[
 \xymatrix{
&E \ar[rd]^k& \\
D\ar[ur]^j&& D\ar[ll]^i 
}
\]
the exactness at each vertex implies that the map $d = j \circ k$ is a
\emph{differential}, in fact $d^2 = (j \circ k) \circ (j \circ k) = j \circ (k \circ j) \circ k = 0$. Now we define a new exact couple called the \emph{derived exact couple} by setting $D'
=\ensuremath{\operatorname{Im}}i$ and $E' = H_d(E) = \ker d /\ensuremath{\operatorname{Im}}d$.

Then, there exists {{\em induced maps\/}} $k' : E'
\longrightarrow D'$, $i' : D' \longrightarrow D'$, $j' : D' \longrightarrow
E'$ \ defined as follows. 
First notice that for any element $[x] \in E'$, where $x \in \ker (d)$, \
we have by the exactness at $E$ of the couple, that $k
(x) \in \ensuremath{\operatorname{Im}} (i)$ (since $x \in \ker (d)$ is
equivalent to 
$(j \circ k) (x) = 0$,
%%sb changes
 which implies that $k (x) \in
\ker (j) =\ensuremath{\operatorname{Im}} (i)$).

Now define
\begin{eqnarray}
\nonumber
  k' ([x]) & = & k (x) \in \ensuremath{\operatorname{Im}}
  (i) = D', \ensuremath{\operatorname{for}}\ensuremath{\operatorname{all}}x
  \in \ker (d), \\
  \label{eqn:defnofderivedcouple}
  i' & = & i \mid_{\ensuremath{\operatorname{Im}}i}, \\
  \nonumber
  j' (i (x)) & = & [j (x)],
  \ensuremath{\operatorname{for}}\ensuremath{\operatorname{all}}i (x) \in
  \ensuremath{\operatorname{Im}}i. 
\end{eqnarray}
It follows easily from these definitions that the $R$-modules $E', D'$,
and the linear maps $k', i', j'$ also form an exact couple. In other
words the following diagram is exact.
\[
\xymatrix{
&E' \ar[rd]^{k'}& \\
D'\ar[ur]^{j'}&& D'\ar[ll]^{i'} 
}
\]

\begin{defn}
  \label{def:derived} This exact couple
  is called the (first) \emph{derived couple} of the
  exact couple $\mathcal{C}$.
\end{defn}

\begin{ex}
Check that all maps in the construction are well defined and that the derived couple is indeed exact.
\end{ex}

It follows that from an exact couple
\[
\xymatrix{
&E_1 \ar[rd]^{k_1}& \\
D_1\ar[ur]^{j_1}&& D_1\ar[ll]^{i_1} 
}
\]
one gets a sequence of exact couples 
\[
\xymatrix{
&E_r \ar[rd]^{k_r}& \\
D_r\ar[ur]^{j_r}&& D_r\ar[ll]^{i_r} 
}
\]
where $D_r = D_{r-1}'$ ,$E_r = E_{r-1}'$ $\forall r \geq 2$.
We thus have proven the 
\begin{theorem} \label{ThmexCouplesSpSeqnonGr}
    Given an exact couple
    \[
\xymatrix{
&E_1 \ar[rd]^{k_1}& \\
D_1\ar[ur]^{j_1}&& D_1,\ar[ll]^{i_1} 
}
\]
we have an induced Spectral Sequence $\{E_r,d_r\}_{r\geq 2}$ (in the sense of definition \ref{SpSeqDef1}) where $E_r = (E_1)^{(r-1)}$ 
and $d_r = j_1^{(r-1)} \circ k_1^{(r-1)} $.
\end{theorem}
We are now ready to prove the 
\begin{theorem} \label{ThmexCouplesSpSeq}
Let
 $ E_1   =  \bigoplus_{p,q \geq 0} E_1^{p,q}$ and
 $ D_1   =  \bigoplus_{p,q \geq 0} D_1^{p,q}  $
be bigraded Abelian Groups and in the exact couple
    \[
\xymatrix{
&E_1 \ar[rd]^{k}& \\
D_1\ar[ur]^{j}&& D_1\ar[ll]^{i} 
}
\]
let $i$ have bidegree $(-1,1)$ , $j$ bidegree $(0,0)$ and $k$ bidegree $(1,0)$.

Then we have an induced (first-quadrant, cohomological type) Spectral Sequence $\{E_r^{p,q},d_r\}_{r\geq 2}$ where $E_r^{p,q} = (E_1^{p,q})^{(r-1)}$ and $d_r = j^{(r-1)} \circ k^{(r-1)} $.
\end{theorem}
\begin{proof}
    We only need to show that the differentials $d_r$ have the right bidegree $(-r,1-r)$. We want to prove that $\deg i^{(r)} = (-1,1)$,  $\deg j^{(r)} = (r-1,1-r)$,  $\deg k^{(r)} = (1,0)$ and finally that $\deg d_{r} = (r,1-r)$. We use induction on $r$.   
    For $r=1$: $d_1=j \circ k$, so $\deg d_1 = \deg j +\deg k= (0,0)+(1,0)=(1,0)$. 
    
    Now suppose $\deg j^{(r-1)}=(r-2,2-r)$, $\deg k^{(r-1)}=(1,0)$, $\deg i^{(r-1)}=(-1,1)$. Since $j^{(r)}(i^{(r-1)}(x)) \defeq [j^{(r-1)}(x)] \in (E_1^{p,q})^{(r)}$ for $x \in (A_1^{p-r+2,q+r-2})^{(r-1)}$, and $i^{(r-1)}(x) \in (A_1^{p-r+1,q+r-1})^{(r)} $, we have $j^{(r)} \colon (A_1^{p-r+1,q+r-1})^{(r)}  \to (E_1^{p,q})^{(r)} $, meaning $\deg j^{(r)} = (r-1,1-r)$. From $k^{(r)}([e]) \defeq k^{(r-1)}(e) $ and $i^{(r)}(x) \defeq i^{(r-1)}(x)$, we see that $i^{(r)}$ and $k^{(r)}$ have the same bidegree as $i$ and $k$ respectively. Hence $\deg d_r = (r-1,1-r) +(1,0) = (r,1-r)$.
\end{proof}





\subsection{Spectral Sequences via filtered complexes} \label{sect_SpSeqfilt}

In this section, we show how a filtration gives rise to a spectral sequence. Let K be a filtered complex, meaning we have a group homomorphism $D\colon K\to K$ s.t. $D^2 = 0$ and  a decreasing filtration $F^\bullet K$, i.e., an ordered family of chain subcomplexes:
\[
\ldots \subseteq F^{n+1}K\subseteq F^nK\subseteq \ldots \subseteq F^0K = K.
\]
We may define a filtration of the homology $H$ of $K$ via the inclusions: 

$F^nH \defeq \ensuremath{\operatorname{Im}}(H_D(F^nK)\longrightarrow H_D(K))$. Thus we have an ordered family of modules:
\[
\ldots \subseteq F^{n+1}H\subseteq F^nH\subseteq \ldots \subseteq F^0H = H.
\]
This filtration is called the induced filtration in (co-)homology.

Denote $B = GK =  \bigoplus\limits_{p \geq 0}\frac{F^pK}{F^{p+1}K}$ the associated graded complex. Define $A = \bigoplus_{p \geq 0}{F^pK} $ to be the differential complex with differential induced by the one of $K$ and we have a map $i \colon A \to A$ induced by the inclusions $i \colon F^{p+1}K \to F^pK$. We then have that $B = \frac{A}{i(A)}$ and we set $j \colon A \to B$ to be the natural projection. We have a short exact sequence 

\begin{equation}\label{eq:ses}
\xymatrix{
  0 \ar[r] & A \ar@{=}[d] \ar[r]^-{i} & A \ar@{=}[d]  \ar[r]^-{j} & B \ar@{=}[d] \ar[r] & 0 \\ 
  0 \ar[r] & \bigoplus\limits_{p \geq 0}{F^pK} \ar[r]_-{i} & \bigoplus\limits_{p \geq 0}{F^pK} \ar[r]_-{j} &  \bigoplus\limits_{p \geq 0}\frac{F^pK}{F^{p+1}K} \ar[r] & 0
}
\end{equation}


If $K$ is a graded differential complex, meaning $K= \bigoplus\limits_{n \geq 0}K^n$ and $D \colon K^n \to K^{n+1}$ is s.t. $D^2 = 0$, then (\ref{eq:ses}) is a short exact sequence of complexes, hence by the Snake Lemma we have a long exact sequence 
\[
\cdots \to   H^n(A) \xrightarrow {i_1} H^n(A) \xrightarrow {j_1}  H^n(B) \xrightarrow {k_1} H^{n+1}(A) \to \cdots
\]
which we can write as an exact couple 

\begin{eqnarray} \label{exactcouple}
  \xymatrix{
&H(B) \ar[rd]^{k_1}& \\
H(A)\ar[ur]^{j_1}&& H(A)\ar[ll]^{i_1} 
} & \defeq & 
  \xymatrix{
&E_1 \ar[rd]^{k_1}& \\
A_1\ar[ur]^{j_1}&& A_1\ar[ll]^{i_1} 
}
\end{eqnarray}

Note that in the case where $K$ is non-graded, we still have this exact couple.

Then, as before, from this exact couple we obtain a sequence of exact couples by taking the derived couples.

Setting $A_1^{p,q} = H^{p+q}(F^pK)$ and $E_1^{p,q} = H^{p+q}\left( \frac{F^pK}{F^{p+1}K} \right) $ , we see that $A_1$ and $E_1$ are bigraded:

\begin{align*}
&A_1 = H(A) = \bigoplus\limits_{p,q \geq 0}H^{p+q}(F^pK) = \bigoplus\limits_{p,q \geq 0}A_1^{p,q} \\
&E_1 = H(B) = \bigoplus\limits_{p,q \geq 0}H^{p+q}\left( \frac{F^pK}{F^{p+1}K} \right) = \bigoplus\limits_{p,q \geq 0}E_1^{p,q}
\end{align*}

and the bidegree of $i_1$, $j_1$, $k_1$ are respectively $(-1,1)$ ,$(0,0)$ ,$(1,0)$ :


\begin{align}
&i_1 \colon A_1^{p,q} = H^{p+q}(F^pK) \longrightarrow H^{p+q}(F^{p-1}K) = A_1^{p-1,q+1} \\
&j_1 \colon A_1^{p,q} = H^{p+q}(F^pK) 	\longrightarrow H^{p+q}\left( \frac{F^pK}{F^{p+1}K} \right) = E_1^{p,q} \\
&k_1 \colon E_1^{p,q} = H^{p+q} \left( \frac{F^pK}{F^{p+1}K} \right) 	\longrightarrow H^{p+q+1}(F^{p+1}K) = A_1^{p+1,q} 
\end{align}
Recall that $k_1$ is the connecting homomorphism coming from the Snake Lemma.
Now we state the most important Theorem about Spectral Sequences.
\begin{theorem}\label{ThmSpSeq}
    Let $K$ be a filtered differential complex (non necessarily graded)
    \[
    (D \colon K \to K  , D^2 = 0  , 
\ldots \subseteq F^{n+1}K\subseteq F^nK\subseteq \ldots \subseteq F^0K = K)
    \]
     and $H_D(K)$ be the filtered differential complex in cohomology. \\
     If $\{F^pK\}_p$ has finite length $l \in \NN$ , then the exact sequence  (\ref{eq:ses}) induces a Spectral Sequence converging to $H_D(K)$.
\end{theorem}
\begin{proof}
    As we pointed out, we have an exact couple (\ref{exactcouple})
    \begin{align*}
     \xymatrix{
&E_1 \ar[rd]^{k_1}& \\
A_1\ar[ur]^{j_1}&& A_1\ar[ll]^{i_1} }
    \end{align*}
    and by Theorem \ref{ThmexCouplesSpSeqnonGr} we have a Spectral Sequence (in the sense of definition \ref{SpSeqDef1}) where $E_1=H(B)$ and $d_1=j_1 \circ k_1$. If the filtration on $K$ has length $l$, then also the induced filtration has length $l$ and the sequence $\{A_r\}_r$ immediately stabilizes, i.e. $A_{l+1}=A_{l+2}= \cdots \eqdef A_\infty$ and since in 
     \begin{align*}
     \xymatrix{
&E_{l+1} \ar[rd]^{k_{l+1}}& \\
A_{l+1}\ar[ur]^{j_{l+1}}&& A_{l+1}\ar[ll]^{i_{l+1}} }
    \end{align*}
    $i_{l+1}$ is an inclusion, $\ker i_{l+1} =0$ and $k_{l+1} \equiv 0$ is the zero map. Then the differentials $d_r$ are all zero for $r$ greater than $l$. This means that the pages of the Spectral Sequence stabilizes also:   $E_{l+1}=E_{l+2}= \cdots \eqdef E_\infty$. In the exact couple 
    \begin{align*}
     \xymatrix{
&E_{\infty} \ar[rd]^{k_{\infty}=0}& \\
A_{\infty}\ar[ur]^{j_{\infty}}&& A_{\infty}\ar[ll]^{i_{\infty}} }
    \end{align*}
    
$A_{\infty}$ is the direct sum of the groups 
\begin{equation} \label{Ainfinity}
    H(K) \supset i_1H(F^1K) \supset i_1i_1H(F^2K) \supset i_1i_1i_1H(F^3K) \supset \cdots \supset i_1^{l+1}H(F^{l+1}K) \supset 0
\end{equation}
and the inclusion $i_\infty$ is as in (\ref{Ainfinity}). Since $E_\infty$ is the quotient of $i_\infty$, it is the direct sum of the successive quotients in $i_\infty$. If we let (\ref{Ainfinity}) be the filtration on $H(K)$, then $E_\infty$ is the associated graded complex of the filtered complex $H(K)$, $E_\infty =G(H(K))$.    
\end{proof}

When the complex $K$ is also graded we have another version of the previous Theorem:
\begin{theorem}\label{ThmSpSeqgraded}
    Let $K$ be a filtered graded differential complex 
    \[
    (D \colon K^n \to K^{n+1}  , D^2 = 0  , 
\ldots \subseteq F^{n+1}K\subseteq F^nK\subseteq \ldots \subseteq F^0K = K)
    \]
    Let $F^pK^n \defeq F^pK \cap K^n$ and let $H_D(K)= \bigoplus_{n \in \Z} H^n_D(K)$ be the graded filtered differential complex in cohomology. 
    
     If for all $n$, $\{F^pK^n\}_p$ has finite length $l(n) \in \NN$ , then the exact sequence  (\ref{eq:ses}) induces a Spectral Sequence converging to $H_D(K)$.
\end{theorem}
\begin{proof}
    The proof is the same as without the grading. Notice also that the spectral sequence converging to $H_D(K)$ is a cohomological type first quadrant spectral sequence $\{E_r^{p,q},d_r\}_{r}$ by Theorem \ref{ThmexCouplesSpSeq} since in the exact couple (\ref{exactcouple}) the maps $i_1, j_1, k_1$ have the right bidegree.
\end{proof}



\subsection{Spectral Sequences via double complexes}

Let $K=\bigoplus\limits_{p,q \in \NN} K^{p,q}$ be a double complex, i.e. a bigraded differential complex with a horizontal differential $d_> \colon K^{p,q} \to K^{p+1,q}$ and a vertical differential $d_{\wedge} \colon K^{p,q} \to K^{p,q+1}$. Define two decreasing filtrations by double complexes:
\begin{align*}
    F^p_{I}K \defeq \bigoplus\limits_{{i \geq p }\atop{ q \geq 0}} K^{i,q} \qquad
     F^p_{II}K \defeq \bigoplus\limits_{{p \geq 0 }\atop{ j \geq q}} K^{p,j}
\end{align*} 
We call the first one the filtration \emph{by columns} and the second one \emph{by rows}. If we let $K^\bullet \defeq \tot(K)^\bullet = \bigoplus\limits_{p+q=\bullet} K^{p,q}$ be the \emph{total complex} of $K^{\bullet,\bullet}$ with the differential $D \colon \tot(K)^n \to \tot(K)^{n+1}$ defined as $D=d_> + (-1)^p d_{\wedge}$, the two filtrations $F^\bullet_I$ and $F^\bullet_{II}$ descend to filtrations of this single graded differential complex $K^\bullet$. They are obtained by considering either all columns to the right of a given column or all rows above a given row :


\begin{minipage}{0.5\textwidth}
\begin{align*}
&\xymatrix@=10pt{
 \cdot & \cdot\ar@{-}[]+C;[rrrddd]+C & \circ \ar@{--}[]+C;[dddd]+C& \circ & \circ \\
 \cdot & \cdot & \bullet & \circ & \circ \\
 \cdot & \cdot & \circ & \bullet\ar[]+C;[r]+C^{d_>}\ar[]+C;[u]+C^{d_{\wedge}} & \circ \\
 \cdot & \cdot & \circ & \circ & \bullet \\
 \cdot & \cdot & \circ & \circ & \circ  \\}
&\\
&\text{Filtration by columns}
\end{align*}
\end{minipage}
\begin{minipage}{0.5\textwidth}
\begin{align*}
&\xymatrix@=10pt{
 \circ & \bullet\ar@{-}[]+C;[rrrddd]+C & \circ & \circ & \circ \\
 \circ & \frm{**}\circ & \bullet \ar[]+C;[r]+C^{d_>}\ar[]+C;[u]+C^{d_{\wedge}}& \circ & \circ \\
 \circ \ar@{--}[]+C;[rrrr]+C & \circ & \circ & \bullet & \circ \\
 \cdot & \cdot & \cdot & \cdot & \cdot \\
 \cdot & \cdot & \cdot & \cdot & \cdot \\
  }
&\\
&\hspace{10pt}\text{Filtration by rows}
\end{align*}
\end{minipage}

\vspace{10pt}These two filtrations give rise to spectral sequences converging to the cohomology of $(tot(K),d_>+ (-1)^p d_\wedge)$ by Thm. \ref{ThmSpSeqgraded}.
Let
\begin{align*}
    H_{d_>}^{p,q}(K) \defeq \frac{\ker(d_> \colon K^{p,q} \to K^{p+1,q})}{\im (d_> \colon K^{p-1,q} \to K^{p,q})} \qquad
     H_{d_\wedge}^{p,q}(K) \defeq \frac{\ker(d_\wedge \colon K^{p,q} \to K^{p,q+1})}{\im (d_\wedge \colon K^{p,q-1} \to K^{p,q})},
\end{align*}
since the differentials commute, i.e. $d_> \circ d_\wedge = d_\wedge \circ d_>$, the groups $ H_{d_>}^{p,\bullet}(K)$ and $  H_{d_\wedge}^{\bullet,q}(K)$ are differential graded abelian groups with induced differentials respectively $d_\wedge $ and $d_>$. So we can also define 
\begin{align*}
    H_{d_>}^{p}H^q_{d_\wedge}(K)  \qquad \text{ and } \qquad 
     H_{d_\wedge}^{q}H^p_{d_>}(K).
\end{align*}
We have the
\begin{theorem} \label{doublecomplexSpSeq}
    Given a double complex $\{K^{\bullet,\bullet}, d_>, d_\wedge \}$, we have two spectral sequences $\{{}_IE_r^{p,q},d_r \}$, $\{{}_{II}E_r^{p,q},d_r \}$ with
    \begin{align*}
    &{}_IE_1^{p,q}=H^{p,q}_{d_\wedge}(K) \qquad  &{}_{II}E_1^{p,q}=H^{p,q}_{d_>}(K) \\
&{}_IE_2^{p,q}=H^p_{d_>}H^{\bullet,q}_{d_\wedge}(K) 
&{}_{II}E_2^{p,q}=H^q_{d_\wedge}H^{p,\bullet}_{d_>}(K)
    \end{align*}
    both converging to $H_D(K^\bullet)$, i.e.
    \begin{align*}
        GH^n_D(K^\bullet) = \bigoplus_{p+q=n} E_\infty^{p,q} \quad \text{ or } \quad E_\infty^{p,q} \simeq \frac{F^p H^{p+q}(K^\bullet)}{F^{p+1} H^{p+q}(K^\bullet)}
    \end{align*}
\end{theorem}
\begin{rem} \label{spseqorientation}
    Note that ${}_IE_r$ has a \emph{rightward orientation}, meaning that the differentials $d_r \colon {}_IE_r^{p,q} \to {}_IE_r^{p+r,q-r+1} $ rotate counterclockwise:

 
\begin{minipage}{0.3\textwidth}
\begin{align*}
&\xymatrix@=0pt{
 \ar@{<-}[]+C;[d]+C\bullet & \bullet & \ar@{<-}[]+C;[d]+C\bullet & \bullet & \bullet \\
 \bullet & \bullet & \bullet & \bullet & \bullet \\
 \bullet & \ar@{<-}[]+C;[d]+C\bullet & \bullet & \bullet & \bullet \\
 \bullet & \bullet & \bullet & \ar@{<-}[]+C;[d]+C\bullet & \bullet \\
 \bullet & \bullet & \bullet & \bullet & \bullet \\
}\\
&\hspace{10pt}(E_0,d_0)=(K^{\bullet,\bullet},d_\wedge)
\end{align*}
\end{minipage}
\begin{minipage}{0.3\textwidth}
\begin{align*}
&\xymatrix@=0pt{
 \bullet & \bullet & \bullet & \bullet & \bullet \\
 \bullet & \ar@{<-}[]+C;[l]+C\bullet & \bullet & \ar@{<-}[]+C;[l]+C\bullet & \bullet \\
 \bullet & \bullet & \bullet & \bullet & \bullet \\
 \bullet & \bullet & \ar@{<-}[]+C;[l]+C\bullet & \bullet & \bullet \\
 \bullet & \bullet & \bullet & \bullet & \ar@{<-}[]+C;[l]+C\bullet \\
}\\
&\hspace{10pt}(E_1,d_1)
\end{align*}
\end{minipage}
\begin{minipage}{0.3\textwidth}
\begin{align*}
&\xymatrix@=0pt{
 \bullet & \bullet & \bullet & \bullet & \bullet \\
 \bullet & \bullet & \bullet & \bullet & \bullet \\
 \bullet & \bullet & \ar@{<-}[]+C;[llu]+C\bullet & \bullet & \ar@{<-}[]+C;[llu]+C\bullet \\
 \bullet & \bullet & \bullet & \bullet & \bullet \\
 \bullet & \bullet & \bullet & \ar@{<-}[]+C;[llu]+C\bullet & \bullet \\
}\\
&\hspace{10pt}(E_2,d_2)
\end{align*}
\end{minipage}

\begin{center}
    
\begin{minipage}{0.3\textwidth}
\begin{align*}
&\xymatrix@=0pt{
 \bullet & \bullet & \bullet & \bullet & \bullet \\
 \bullet & \bullet & \bullet & \bullet & \bullet \\
 \bullet & \bullet & \bullet & \bullet & \ar@{<-}[]+C;[llluu]+C\bullet \\
 \bullet & \bullet & \bullet & \ar@{<-}[]+C;[llluu]+C\bullet & \bullet \\
 \bullet & \bullet & \bullet & \bullet & \bullet \\
}\\
&\hspace{10pt}(E_3,d_3)
\end{align*}
\end{minipage}
\begin{minipage}{0.3\textwidth}
\begin{align*}
&\xymatrix@=0pt{
 \bullet & \bullet & \bullet & \bullet & \bullet \\
 \bullet & \bullet & \bullet & \bullet & \bullet \\
 \bullet & \bullet & \bullet & \bullet & \bullet \\
 \bullet & \bullet & \bullet & \bullet &  \ar@{<-}[]+C;[lllluuu]+C\bullet \\
 \bullet & \bullet & \bullet & \bullet & \ar@{<-}[]+C;[lllluuu]+C\bullet \\
}\\
&\hspace{10pt}(E_4,d_4)
\end{align*}
\end{minipage}
\end{center}
Meanwhile, the spectral sequence  ${}_{II}E_r$ has a \emph{leftward orientation}, meaning that the differentials $d_r \colon {}_IE_r^{p,q} \to {}_IE_r^{p-r+1,q+r} $ rotate clockwise:


\begin{minipage}{0.3\textwidth}
\begin{align*}
&\xymatrix@=0pt{
 \bullet & \bullet & \bullet & \bullet & \bullet \\
 \bullet & \ar@{<-}[]+C;[l]+C\bullet & \bullet &\bullet & \bullet \\
 \bullet & \bullet & \bullet & \bullet &  \ar@{<-}[]+C;[l]+C\bullet \\
 \bullet & \bullet & \bullet & \ar@{<-}[]+C;[l]+C\bullet & \bullet \\
 \bullet & \bullet & \bullet & \bullet & \ar@{<-}[]+C;[l]+C\bullet \\
}\\
&\hspace{10pt}(E_0,d_0)=(K^{\bullet,\bullet},d_>)
\end{align*}
\end{minipage}
\begin{minipage}{0.3\textwidth}
\begin{align*}
&\xymatrix@=0pt{
 \ar@{<-}[]+C;[d]+C\bullet & \bullet & \bullet & \bullet & \bullet \\
 \bullet & \bullet & \bullet & \ar@{<-}[]+C;[d]+C\bullet & \bullet \\
 \bullet & \bullet & \ar@{<-}[]+C;[d]+C\bullet & \bullet & \bullet \\
 \bullet & \bullet & \bullet & \ar@{<-}[]+C;[d]+C\bullet & \bullet \\
 \bullet & \bullet & \bullet & \bullet & \bullet \\
}\\
&\hspace{10pt}(E_1,d_1)
\end{align*}
\end{minipage}
\begin{minipage}{0.3\textwidth}
\begin{align*}
&\xymatrix@=0pt{
 \bullet & \bullet & \ar@{<-}[]+C;[rdd]+C\bullet & \bullet & \bullet \\
 \bullet & \ar@{<-}[]+C;[rdd]+C\bullet & \bullet & \bullet & \bullet \\
 \bullet & \bullet & \ar@{<-}[]+C;[rdd]+C\bullet & \bullet & \bullet \\
 \bullet & \bullet & \bullet & \bullet & \bullet \\
 \bullet & \bullet & \bullet & \bullet & \bullet \\
}\\
&\hspace{10pt}(E_2,d_2)
\end{align*}
\end{minipage}

\begin{center}
    
\begin{minipage}{0.3\textwidth}
\begin{align*}
&\xymatrix@=0pt{
 \ar@{<-}[]+C;[rrddd]+C\bullet & \bullet & \ar@{<-}[]+C;[rrddd]+C\bullet & \bullet & \bullet \\
 \bullet & \ar@{<-}[]+C;[rrddd]+C\bullet & \bullet & \bullet & \bullet \\
 \bullet & \bullet & \bullet & \bullet & \bullet \\
 \bullet & \bullet & \bullet & \bullet & \bullet \\
 \bullet & \bullet & \bullet & \bullet & \bullet \\
}\\
&\hspace{10pt}(E_3,d_3)
\end{align*}
\end{minipage}
\begin{minipage}{0.3\textwidth}
\begin{align*}
&\xymatrix@=0pt{
  \ar@{<-}[]+C;[rrrdddd]+C\bullet & \ar@{<-}[]+C;[rrrdddd]+C\bullet & \bullet & \bullet & \bullet \\
 \bullet & \bullet & \bullet & \bullet & \bullet \\
 \bullet & \bullet & \bullet & \bullet & \bullet \\
 \bullet & \bullet & \bullet & \bullet & \bullet \\
 \bullet & \bullet & \bullet & \bullet & \bullet \\
}\\
&\hspace{10pt}(E_4,d_4)
\end{align*}
\end{minipage}
\end{center}



\end{rem}
\begin{proof}
    We give a proof for $\{{}_IE_r^{p,q},d_r \}$, the other case will follow from symmetry. 
    The filtration $F^p_I$ on $K^{\bullet,\bullet}$ induces a filtration on $\tot(K)$:
    \[
    F^p_I (\tot(K))^n = \bigoplus_{i \geq p} K^{i,n-i}.
    \]
    This filtration has finite length for all $n$. In fact:
    \[
    \cdots \supset F^n_I (\tot(K))^n = K^{n,0} \supset F_I^{n+1} (\tot(K))^n = 0
    \]
    So by Thm. \ref{ThmSpSeqgraded} we have a spectral sequence converging to $H_D(\tot(K))$:
    \[
    {}_IE_1^{p,q} \defeq H_D^{p+q} \left( \frac{F^p_I \tot K}{F^{p+1}_I \tot K} \right) \Rightarrow {}_IE_{\infty}^{p,q} = \frac{F^p_I H_D^{p+q}(\tot K )}{F^{p+1} H_D^{p+q}( \tot K)}
    \]
    First we show ${}_IE_1^{p,q} \cong H^{p,q}_{d_\wedge} (K)$. This follows since $D=d_> + (-1)^p d_\wedge$ and $d_> ( F^p_I \tot K) \subset F^{p+1}_I \tot K $. In fact we have
    \[
     \left( \frac{F^p_I \tot K}{F^{p+1}_I \tot K} \right)^{p+q} \cong K^{p,q} \qquad \text{ and } \qquad D=(-1)^pd_\wedge \quad \text{ on } \quad B= \bigoplus_p  \left( \frac{F^p_I \tot K}{F^{p+1}_I \tot K} \right)
    \]
    Now we show what the $E_2$-page looks like.
    Recall $A \defeq \bigoplus_p F^p_I \tot K$ and that $k_1 \colon H^n(B) \to H^n(A)$ is given by:
    
\begin{equation*}
\begin{tikzcd}
 & \vdots & \vdots & \vdots & \\
0 \arrow[r] & F^{p+1}_I (\tot K)^{n+1} \arrow[r,"i","(3)"']   \arrow[u] & F^{p}_I (\tot K)^{n+1} \arrow[r, two heads]  \arrow[u] & \frac{F^{p}_I (\tot K)^{n+1}}{F^{p+1}_I (\tot K)^{n+1}} \arrow[r] \arrow[u] & 0 \\
0 \arrow[r] & F^{p+1}_I (\tot K)^{n} \arrow[r]   \arrow[u] & F^{p}_I (\tot K)^{n} \arrow[r,"(1)", two heads]  \arrow[u,"(2)", "D"'] & \frac{F^{p}_I (\tot K)^{n}}{F^{p+1}_I (\tot K)^{n}} \arrow[r] \arrow[u] & 0 \\
0 \arrow[r] & F^{p+1}_I (\tot K)^{n-1} \arrow[r]   \arrow[u] & F^{p}_I (\tot K)^{n-1} \arrow[r, two heads]  \arrow[u] & \frac{F^{p}_I (\tot K)^{n-1}}{F^{p+1}_I (\tot K)^{n-1}} \arrow[r] \arrow[u] & 0 \\
 & \vdots \arrow[u] & \vdots \arrow[u] & \vdots \arrow[u] &
\end{tikzcd}
\end{equation*}
We choose a representative $x \in F^{p}_I (\tot K)^{n} $ for $[x] \in  \frac{F^{p}_I (\tot K)^{n}}{F^{p+1}_I (\tot K)^{n}} $. Take $D(x) \in F^{p}_I (\tot K)^{n+1} $, then $k_1([x])$ is the inverse under $i$ of $D(x)$.
Now, if $x$ represents an element of
\[
{}_IE_1^{p,q} \defeq H_D^{p+q} \left( \frac{F^p_I \tot K}{F^{p+1}_I \tot K} \right) \cong H^{p,q}_{d_\wedge} (K)
\]
we have $d_\wedge(x)=0$, so $D(x)=d_>(x)$ and $k_1([x])=[d_>(x)]$. Then the differential $d_1=j_1 \circ k_1 \colon {}_IE_1^{p,q} \lra {}_IE_1^{p+1,q} $ is given by $d_>$ on $H^{p,q}_{d_\wedge}$. Hence
\[
 {}_IE_2^{p,q} \cong H^p_{d_>}H^{\bullet,q}_{d_\wedge}(K) 
\]
which closes the proof.
\end{proof}




\subsection{Hypercohomology}\label{HypCoh}

Let 
\[\cdots \stackrel{d}{\longrightarrow} A^0\stackrel{d}{\longrightarrow} A^1 \stackrel{d}{\longrightarrow} A^2 \stackrel{d}{\longrightarrow} \cdots\]
be a complex of sheaves on a space $X$. Recall that the \emph{hypercohomology} of the complex $(A^\bullet,d)$ is defined by choosing an acyclic resolution of $A^\bullet$ by a double complex $(I^{\bullet,\bullet},d,d')$, i.e. a diagram with exact rows and columns
\begin{center}
\centerline{
\xymatrix@C=30pt
{
 & \vdots&\vdots&\vdots\\
\cdots\ar[r] & I^{0,1} \ar_d[r]\ar_{d'}[u]& I^{1,1} \ar_d[r]\ar_{d'}[u]& I^{2,1} \ar_d[r]\ar_{d'}[u]&\cdots\\
\cdots\ar[r] & I^{0,0} \ar_d[r]\ar_{d'}[u]& I^{1,0} \ar_d[r]\ar_{d'}[u]& I^{2,0} \ar_d[r]\ar_{d'}[u]&\cdots\\
\cdots\ar[r] & A^0 \ar_d[r]\ar_{d'}[u]& A^1 \ar_d[r]\ar_{d'}[u]& A^2 \ar_d[r]\ar_{d'}[u]&\cdots\\
 & 0 \ar[u] & 0 \ar[u] & 0 \ar[u]
}}
\end{center}
and $H^n(X,I^{i,j})=0$ for $n\ge 0$. Such a resolution always exists. Let $I^n :=\bigoplus_{i+j=n} I^{i,j}$ denote the total complex with differential $\delta=d+(-1)^id'$, then the $n$th hypercohomology of $(A^\bullet,d)$ is defined to be the $n$th cohomology of the complex of global sections of $I^{\bullet}$,
\[\HH^n(X,A^\bullet):=H^n_\delta \Gamma(X,I^\bullet),\]
and this definition is independent of the choice of $I^{\bullet,\bullet}$.

Next assume that we have a decreasing filtration $F$ on $(A^\bullet,d)$ turning it into a filtered complex, i.e. a decreasing filtration 
\[\cdots \subset F^2A^k \subset F^1A^k \subset F^0A^k=A^k\] 
on each $A^k$, such that $d$ preserves the filtration  $d\colon F^pA^k\ra F^pA^{k+1}$. Replacing $I^{\bullet,\bullet}$ if necessary, we may assume that the filtration $F$ lifts to the resolution, in other words, that we have a filtration 
\[ I^{i,j}=F^0I^{i,j} \supset F^1I^{i,j} \supset F^2I^{i,j} \supset \cdots\] 
on the double complex $I^{\bullet,\bullet}$ that is compatible with differentials, such that $F^iI^{\bullet,\bullet}$ is an acyclic resolution of $F^iA^\bullet$. As before, we denote the total complex of $F^iI^{\bullet,\bullet}$ by $F^iI^{\bullet}$ and denote its differential by $\delta$. 

We may use this filtration $F$ to define a filtration on the hypercohomology of $A^\bullet$ as follows. The embedding $F^iI^{\bullet}\subseteq I^{\bullet}$ induces a map of hypercohomologies
\[\HH^n(X,F^pA^\bullet)\lra\HH^n(X,A^\bullet).\]
We may then simply define $F^p\HH^n(X,A^\bullet)$ to be the image of this map.

Now suppose that the filtration $F^pA^{\bullet}$ is bounded, so that for each $k$, there exists a $p$ with $F^pA^k = 0$. Then
applying Thm. \ref{ThmSpSeq} to the filtered complex $(\Gamma(X,I^\bullet),F^\bullet)$, we have:

\begin{theorem}\label{thm:specseq} \textup{\cite[Thm. 8.21]{Voi07}}   There exist complexes 
\[(E^{p,q}_r,d_r), \quad \mbox{with differentials } \, d_r\colon E^{p,q}_r \rightarrow E^{p+r,q-r+1}_r\]
which satisfy the following conditions:
\begin{enumerate}
\item $E_0^{p,q} = \Gamma(X,\,\Gr_F^p I^{p+q}) := \Gamma(X,\,F^p I^{p+q}/F^{p+1} I^{p+q})$ and $d_0$ is induced by $\delta$.
\item $E_1^{p,q} = H^{p+q}_\delta(\Gamma(X,\,\Gr_F^p I^{p+q})) = \HH^{p+q}(X,\Gr_F^p A^\bullet)$
\item $E^{p,q}_{r+1}$ can be identified with the cohomology of $(E_r^{p,q},d_r)$, i.e. with
\[\frac{\mathrm{ker}(d_r\colon E^{p,q}_r \to E^{p+r,q-r+1}_r)}{\im(d_r\colon E^{p-r,q+r-1} \to E^{p,q}_r)}.\]
\item For $p + q$ fixed and $r$ sufficiently large,
\[E^{p,q}_r = \Gr_F^p H^{p+q}(\Gamma (I^\bullet)) = \Gr_F^p \HH^{p+q}(X,A^\bullet)\]
\end{enumerate}
\end{theorem}

We note here that the exactness of $\Gamma$ on acyclic objects implies that $\Gamma$ commutes with $\Gr_F$, so in part 1 of the above theorem we have
\[\Gamma\left(X,\Gr_F^p I^{p+q}\right) = \frac{\Gamma(X,\,F^p I^{p+q})}{\Gamma(X,\,F^{p+1} I^{p+q})}.\]

The series of complexes $(E^{p,q}_r,d_r)$ is called the \emph{spectral sequence} associated to the filtered complex $(A^{\bullet},F)$. Part 3 of Thm. \ref{thm:specseq} says that this spectral sequence \emph{converges} to $\HH^{p+q}(X,A^\bullet)$. This group is often denoted by $E^{p+q}_{\infty}$, and its graded piece $\Gr_F^p \HH^{p+q}(X,A^\bullet)$ denoted by $E^{p,q}_{\infty}$. We write
\[ E_0^{p,q}=\Gamma(X,\Gr_F^p I^{p+q}) \Rightarrow \HH^{p+q}(X,A^\bullet),\]
which should be read as ``the spectral sequence with $E_0^{p,q}=\Gamma(X,\Gr_F^p I^{p+q})$ converges to $ \HH^{p+q}(X,A^\bullet)$''. Note here that it is also not uncommon to see a spectral sequence defined using $E_r^{p,q}$, for some $r>0$, on the left hand side instead of $E_0^{p,q}$.

We say that the spectral sequence associated to a filtered complex $(A^\bullet, F)$ \emph{degenerates at $E_r$} if $d_k = 0$ for all $k \geq r$. For such $r$, from Thm. \ref{thm:specseq} we obtain 
\[ E_r^{p,q} = E_\infty^{p,q} = \Gr_F^p \HH^{p+q}(X,A^\bullet).\]


\end{document}