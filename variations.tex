\documentclass[../main.tex]{subfiles}

\title{Capitolo 3 - Variazioni di strutture di Hodge}
\author{Edoardo Manini}
\date{Febbraio 20, 2025}

\begin{document}
\ifSubfilesClassLoaded{
\maketitle
\tableofcontents
}{}


Now we want to give a good definition of what it means to have (polarized) Hodge structures that vary in families of compact K\"{a}hler manifolds. To do this we define the concept of a \emph{polarized variation of Hodge structure} and the \emph{period map} associated to it. Motivated by the compact K\"{a}hler case, we will then be able to give a more general definition of (integral polarized) \emph{variation of Hodge structure} $\Z$-PVHS. 

We are interested in studying this theory to understand what happens, as we deform a K\"{a}hler manifold, to the Hodge structure on its cohomology.


\subsection{Local Period Domain} \label{sect:locmap}

Consider the following set up. 
Let $f\colon \calX \to B$ be a proper holomorphic submersion onto a connected complex manifold $B$, whose fibres $X_b$ are compact K\"{a}hler manifolds for all $b \in B$. If $0 \in B$ is a reference point, we say that $ \calX $ is a \emph{family} of deformations of the fibre $X_0$.
Assume that there exists $\omega \in H^2(\calX,\Z)$ such that, for each $b \in \Delta$, the restriction $\omega_b := \omega|_{X_b}$ is an integral K\"{a}hler class. This happens for example if  $f\colon \calX \to B$ is a family of projective manifolds $f\colon \calX \subset \PP^n \times B \to B$ and $f$ is the restriction to $\calX$ of the projection to $B$, since $\calX$ has the K\"{a}hler form coming from projective space, which is integral.
This induces a polarized Hodge structure on the cohomology of the fibres $H^n(X_b,\Z)$ as we have seen before, which varies with $b \in B$. This is the main example of a \emph{variation of polarized Hodge structure} which forms the ispiration to naturally give a general definition for such objects.

If $B$ is a contractible manifold, we see that smoothness of $f$ implies that the fibres $X_b$ are all diffeomorphic, and Ehresmann's theorem \cite[Thm. 9.3]{Voi07} shows that $\calX$ is diffeomorphic to $X_b \times B$ over $B$, for any $b \in B$. 
Therefore, there is a unique isomorphism $H^n(X_b,\Z) \cong H^n(X_{b'},\Z)$ for any $b,b'\in B$. 
Thus, we can define $H_{\Z} \defeq H^n(X_b,\Z)/\mathrm{torsion}$ and $H_\CC \defeq H_{\Z} \otimes_{\Z} \C \cong H^n(X_b,\C)$, and these definitions do not depend on the choice of $b \in B$. 
In the same manner as before, $\omega$ induces a bilinear form $Q$ on $H_{\Z}$, which can be extended to $H_\CC$ by linearity.

However, there is no guarantee that the isomorphisms $H^n(X_b,\C) \cong H^n(X_{b'},\C)$ preserve the Hodge subspaces and in fact in general they \emph{do not} preserve the Hodge decompositions; but instead, the Hodge decomposition of $H^n(X_b,\C)$ varies in a continuous way with $b \in B$.
Furthermore, even though the subspaces $H^{p,q}$ vary with $b \in B$, it can be proven that their dimensions $h^{p,q} := \dim(H^{p,q})$ remain fixed \cite[Ch. 9]{Voi07}.
To see this, note that the sum of Hodge numbers $(p,q)$ for $p + q = n$ is a constant number, the $n$-th Betti number of the fiber, which is constant because the family is differentiably locally trivial. To show that each of the numbers $h^{p,q}(X_b)$ is constant, it is therefore sufficient to show that they satisfy an upper-semicontinuity property \cite[Cor. 9.19]{Voi07} :
\[
h^{p,q}(X_b) \leq h^{p,q}(X_0), \text{ for } |b - 0| < \epsilon 
\]
One can also think of this variation of the Hodge decomposition of the cohomology of the fibres of the family as a variation of complex structure on a fixed manifold since the $X_b$ are all diffeomorphic.


In an attempt to classify all possible Hodge decompositions on a given complex vector space, We now define a moduli space for these Hodge decompositions with fixed type.

\begin{defn} \label{Ddefn} Fix a type $\underline{h} = (h^{k,0}, \dots , h^{0,k})$ (with $h^{p,q}=h^{q,p}$) and a pair $(H_\Z, Q)$, $H_\Z$ a lattice and $Q$ an alternating/symmetric nondegenerate bilinear form depending on the partiy of $k$. 
The \emph{local period domain} is 
 \begin{align*}
 \calD =  \left\{ \{H^{p,q}\}  \ \middle\vert \begin{array}{l}
    H^{p,q} \subset H_\CC \quad \text{ s.t.} \quad H_\CC = \bigoplus_{p+q=n} H^{p,q},  \\
    \dim (H^{p,q}) = h^{p,q} \quad \textrm{and} \quad Q \text{ satisfies } \eqref{2HdgRiem} \text{ and } \eqref{3HdgRiem}
  \end{array}\right\}
  \end{align*}
the set of all collections of subspaces $\{H^{p,q}\}$ of $H_\CC$ such that $H_\CC = \bigoplus_{p+q=n} H^{p,q}$ and $\dim (H^{p,q}) = h^{p,q} \left( =h^{q,p}= \dim (H^{q,p}) \right)$, on which $Q$ satisfies the Hodge-Riemann bilinear relations \eqref{2HdgRiem} and \eqref{3HdgRiem}.
\end{defn}


\begin{rem} In terms of filtrations, $\calD$ may be defined as the set of all filtrations 
\[ H_\CC =F^0 \supset F^1 \supset \cdots \supset F^n \supset \{0\},\]
with $\dim(F^p) = h^{n,0} + \cdots + h^{p,n-p}$, on which $Q$ satisfies \eqref{2HdgRiemprime}$^\prime$ and \eqref{3HdgRiemprime}$^\prime$.
\end{rem}

Observe that a filtration satisfying  \eqref{2HdgRiemprime}$^\prime$ and \eqref{3HdgRiemprime}$^\prime$ is also a Hodge filtration, i.e. it satisfies $F^p \cap \overline{F^{q+1}} = 0$. Indeed, if $x \in F^p  \cap \overline{F^{q+1}} $, then $\overline{x} \in \overline{ \overline{F^{q+1}}} = F^{q+1}$, hence by \eqref{2HdgRiemprime}$^\prime$, $Q(x,\overline{x})=0 $, but $x \in F^p \cap \overline{F^{q}} $, so by \eqref{3HdgRiemprime}$^\prime$, $x=0$.

$\calD$ is the space of all Hodge structures of the given type  $\underline{h}$ polarized by $Q$. One can realize $\calD$ as a (real) manifold (see \cite[Sect. 4.3]{CMSP03}). Indeed defining the flag variety
\[
Fl(\underline{h}, H_\C) = \left\{ \text{ desc. filtrations } F^\bullet \text{ on } H_\C \text{ s.t. } h^{p,q} = \dim F^p / F^{p+1} \right\},
\]
it naturally has a structure as a projective algebraic variety since it is a subvariety of products of Grassmanians $Gr(f^0,H_\CC) \times \cdots \times Gr(f^n,H_\C)$ where $f^p = h^{n,0} + \cdots + h^{p,n-p} $.
The orthogonal flag variety $\check{\calD} \defeq OFl(\underline{h},H_\C) \subseteq Fl(\underline{h}, H_\C)$ is the closed algebraic subvariety obtained imposing the (closed) condition $(F^p)^\perp =F^{k-p+1}$ for all $p$, i.e. only the first bilinear relation \eqref{2HdgRiemprime}$^\prime$ but not necessarily the second \eqref{3HdgRiemprime}$^\prime$.  $\check{\calD}$ is called the \emph{compact dual} of $\calD$ and can be equivalently defined as the set of all collections of subspaces $\{H^{p,q}\}$ as in Defn. \ref{Ddefn} that satisfy the first Hodge-Riemann bilinear relation  \eqref{2HdgRiem} but not necessarily the second  \eqref{3HdgRiem}.


Finally, imposing the second Hodge-Riemann bilinear relation \eqref{3HdgRiem} gives an open condition $i^{p-q}(-1)^{\frac{k(k-1)}{2}} Q(v, \bar v) > 0$ for all $0 \neq v \in F^p \cap \overline{F^q}$.
In conclusion, the period domain $\calD$ is embedded as an open subvariety in $\check{\calD}$, hence in particular $\calD$ is a complex manifold.

Assuming $B$ is simply connected, there is a well-defined morphism $\phi\colon B \to \calD$, where $\phi$ takes $b \in B$ to the point in $\calD$ corresponding to the Hodge decomposition of $H^n_{\text{prim}}(X_b,\C)$. This morphism is called the  \emph{local period mapping}. 

The bundles $\calH^{p,q}_\CC \defeq  \bigcup_{b \in B} H^{p,q}(X_b,\C) $ are $\calC^\infty$-subbundles of the Hodge bundle $\calH^n_\C = \bigcup_{b \in \Delta} H^n(X_b,\C) = R^nf_*\C_{\calX}$ of the family but are in general \emph{not} holomorphic subbundles.

Hence we consider other subbundles, namely the ones induced by the Hodge filtration $\{F^p_b\}$ on $H^n(X_b,\C)$. They form a filtration of the vector bundle (or locally free sheaf of $\calO_B$-modules) $\calH^n = \calH^n_\C \otimes \calO_B$ by holomorphic subbundles. We call them \emph{Hodge subbundles} and denote them $\calF^p\calH^n \subset \calH^n$.
Equipping $\calH^n$ with the flat connection $\nabla$, called \emph{Gauss-Manin connection}, we find that these bundles have the following property known as Griffiths transversality \cite[Sect. 4.4]{CMSP03}:
\begin{align*}
\nabla \calF^p\calH^n &\subset \calF^{p-1}\calH^n \otimes \Omega^1_B && \mathrm{(\emph{Griffiths transversality})}.
\end{align*}
It can also be shown that the local period mapping $\phi$ is holomorphic \cite[Thm. 4.5.6]{CMSP03}.

To illustrate these ideas, we now compute the local period domains for the two examples studied in the previous section.

\begin{es} Let $E$ be an elliptic curve. 
The polarized Hodge structure on the first cohomology $H_\ZZ = H^1(E,\Z)$ has Hodge numbers $h^{1,0} = h^{0,1} = 1$. Hence we are interested in polarized hodge strcture of weight $1$ and type $(1,1)$ on the lattice $\Z \cdot \alpha \oplus  \Z \cdot \beta$ with the standard symplectic form $\alpha\cdot \beta = 1$. The Hodge filtration is
\[ H_\CC = F^0 \supset F^1 \supset \{0\},\]
where $F^1 = H^{1,0}$. We see that $\calD$ is the set of all filtrations $\C^2 \supset F^1 \supset \{0\}$ with $\dim(F^1) = 1$, on which $Q$ satisfies the conditions \eqref{2HdgRiemprime}$^\prime$ and \eqref{3HdgRiemprime}$^\prime$.

We see that giving a point in  $\calD$ is equivalent to give $ F^1=H^{1,0}$, a point in $Gr(1,\CC^2) = \PP^1_\CC$. 
To do this it suffices to give $\lambda \in H_\CC$ that spans $F^1$. 
Furthermore, in order for this to define a Hodge structure we need that $H^{1,0} \cap \overline{H^{1,0}} = {0}$, i.e. $H^{1,0} \in Gr(1,\CC^2) \setminus Gr(1,\RR^2) =  \PP^1_\CC \setminus  \PP^1_\RR = \mathfrak{H} \cup -\mathfrak{H}$.

The Hodge-Riemann bilinear relations \eqref{2HdgRiemprime}$^\prime$ and \eqref{3HdgRiemprime}$^\prime$ become
\begin{enumerate}[(1')]\addtocounter{enumi}{1}\setlength{\itemindent}{\parindent}
\item $Q(\lambda,\lambda) = 0$,
\item $i Q(\lambda,\overline{\lambda}) > 0$.
\end{enumerate}
We describe $\lambda$ in terms of the canonical basis as $\lambda = z_1\alpha + z_2 \beta$, for $z_1,z_2 \in \C$. The relations become
\begin{enumerate}[(1')]\addtocounter{enumi}{1}\setlength{\itemindent}{\parindent}
\item \label{new2prime} $z_1z_2 - z_2z_1 = 0$,
\item \label{new3prime} $i (z_1\overline{z_2} - z_2\overline{z_1} ) > 0$.
\end{enumerate}
\eqref{new2prime}$^\prime$ is empty in this case and \eqref{new3prime}$^\prime$ implies $z_1 \neq 0$. Thus we can replace $\lambda$ so that $\lambda = \alpha + z_2 \beta$ (i.e. set $z_1 = 1$). Then (3') says $i (\overline{z_2} - z_2 ) > 0$, i.e. $i (-2i \mathfrak{Im} (z_2)) > 0$, hence
\begin{enumerate}[(1')]\addtocounter{enumi}{2}\setlength{\itemindent}{\parindent}
\item $\mathfrak{Im} (z_2) > 0$.
\end{enumerate}
Because identifying $z_2$ is the same as identifying $\lambda$, which uniquely determines $F^1$, we find that the local period domain for elliptic curves is the complex upper half-plane
\[
\calD \cong \mathfrak{H} \defeq \{z \in \C \mid \mathfrak{Im}(z) > 0\}.
\]
\end{es}

\begin{es} 
We now turn to the a curve $C$ of genus $g \geq 1$. The polarized Hodge structure on the first cohomology $H_\Z = H^1(C,\Z)$ has Hodge numbers $h^{1,0} = h^{0,1} = g$. 
So we are interested in the space of all polarized weight $1$ Hodge structures of type $(g, g)$ on $(V, Q)$, $V \simeq \Z^{2g}$ and $Q$ the standard symplectic form given on a standard symplectic basis $\{ \alpha_i, \beta_i \}$ by  $Q(\alpha_i,\alpha_j) = Q(\beta_i,\beta_j) = 0$ for all $i,j$ and $Q(\alpha_i,\beta_j) = \delta_{ij}$.

As in the case of the elliptic curve, the Hodge filtration is
\[ H_\CC = F^0 \supset F^1 \supset \{0\},\]
where $F^1 = H^{1,0}$. We see that $\calD$ is the set of all filtrations $\C^{2g} \supset F^1 \supset \{0\}$ with $\dim(F^1) = g$, on which $Q$ satisfies the Hodge-Riemann bilinear relations \eqref{2HdgRiemprime}$^\prime$ and \eqref{3HdgRiemprime}$^\prime$.

To give a point in $\calD$ is equivalent to give a basis $\{\lambda_1,\ldots,\lambda_g\}$ for the subspace $F^1$. Write $(\lambda_1,\ldots,\lambda_g)$ in terms of the canonical basis $\{\alpha_i,\beta_j\}$ as $(\lambda_1,\ldots,\lambda_g) = (\alpha_1,\ldots,\alpha_g,\beta_1,\ldots,\beta_g)Z$, where $Z = \left(\begin{array}{c} Z_1 \\ Z_2 \end{array} \right)$ is a $2g \times g$ complex matrix. Relations \eqref{2HdgRiemprime}$^\prime$ and \eqref{3HdgRiemprime}$^\prime$ may then be written in terms of the $g \times g$ matrices $Z_1$ and $Z_2$ as follows:
\begin{enumerate}[(1')]\addtocounter{enumi}{1}\setlength{\itemindent}{\parindent}
\item \label{new2primehere} $Z_1^TZ_2 - Z_2^TZ_1 = 0$, i.e. the matrix $Z_1^TZ_2$ is symmetric, and
\item \label{new3primehere} $i (Z_1^T\overline{Z_2} - Z_2^T\overline{Z_1} )$ is a positive-definite matrix.
\end{enumerate}

In particular, \label{new3primehere}$^\prime$ here implies that $Z_1$ is invertible. We may therefore replace $(\lambda_1,\ldots,\lambda_g) \mapsto (\lambda_1,\ldots,\lambda_g)Z_1^{-1}$, so that $Z$ becomes $Z = \left(\begin{array}{c} I_g \\ Z_2 \end{array} \right)$. Then the conditions above become
\begin{enumerate}[(1')]\addtocounter{enumi}{1}\setlength{\itemindent}{\parindent}
\item $Z_2^T - Z_2 = 0$, i.e. the matrix $Z_2$ is symmetric, and
\item $\mathfrak{Im} (Z_2)$ is positive definite.
\end{enumerate}
Thus, we find that in this case the local period domain for genus $g$ curves is the \emph{Siegel upper half-space of degree $g$}
\[\calD \cong \mathfrak{H}_g := \{Z \in M_{g\times g}(\C) \mid Z\ \mathrm{is}\ \mathrm{symmetric}\ \mathrm{and}\ \mathfrak{Im}(Z)\ \mathrm{is}\ \mathrm{positive}\ \mathrm{definite}\}.\]

This is a Hermitian symmetric space of non-compact type since it can be realized as the quotient $\operatorname {Sp}(2g,\RR) / U(g)$ of a simple Lie group by a (maximal in this case) compact subgroup. Here we denote by $\operatorname {Sp}(2g,\RR) = \left\{ M\in M_{2g\times 2g}(\RR):M^{\mathrm {T} } J M=J \ \right\}$ the classic symplectic group where 
\[ 
J =\left( \begin{array}{cc} 0 & I_g \\ -I_g & 0 \end{array} \right),
\] and we denote by $U(g) = \left \{
 M\in M_{g\times g}(\CC): M^* =M^{-1} \right\}$ the unitary group. This can be seen by showing that $\operatorname {Sp}(2g, \RR)$ acts transitively on $\mathfrak{H}_g$ and that the isotropy group at $i1_g$ is $U(g)$. For an introduction to symmetric spaces see \cite{Hel64}.
\end{es}

\begin{rem}
    It can actually be proved that all period domains are of this type. Namely we have a theorem giving a description of all  period domains of which we only give the statement here. For a proof see  \cite[Thm. 4.4.4]{CMSP03}.
    
    \begin{theorem}
        Let $\calD$ be a period domain for weight $n$ Hodge structures $(H_\ZZ,H^{p,q})$ with Hodge type  $\underline{h} = (h^{n,0}, \dots , h^{0,n})$. Then $\calD= G / V$ for $G$ a simple real Lie group.
        In details,
        \begin{enumerate}
            \item  For odd weight $n = 2m + 1$ we have
            \begin{align*}
                G &\simeq Sp(g, \RR), \quad \dim H_\CC = 2g, \\
V &= \prod_{p\leq m} U(h^{p,q}).
            \end{align*}
In this case $\calD$ is connected and non-compact.
\item For even weight $n = 2m$ we have
\begin{align*}
G &\simeq SO(s, t), \qquad s =\sum_{p \text{  even} }h^{p,q}, t =\sum_{p \text{ odd} } h^{p,q},  \\
V &= \prod_{p<m} U(h^{p,q}) \times SO(h^{m,m}).
\end{align*}
In this case, if $s, t > 0$, the domain $\calD$ consists of two isomorphic connected components and if $s = 0 \text{ or } t = 0$, the domain $\calD$ is connected and compact.
        \end{enumerate}
    \end{theorem}
\end{rem}


\subsection{Global Period domain} \label{sect:globmap}

We now want to study families $f\colon \calX \to B$ over a base that is not contractible. 
If our base $B$ is a not necessarily simply connected complex manifold, then the isomorphism $H^n(X_b,\Z) \cong H^n(X_{b'},\Z)$ for $b,b' \in B$ is not necessarily unique.
This means that the period mapping is no longer well-defined.
Indeed there is no longer a unique homotopy class of diffeomorphisms relating the fibers $X_b$ and $X_0$ since different homotopy classes of paths in $B$ induce possibly different homotopy classes of diffeomorphisms and therefore possibly different isomorphisms of cohomology groups. As a consequence there is no longer a well-defined flag for each fiber but the flag is defined only up to the action of the fundamental group.
In order to compensate for this, we must quotient the period domain $\calD$ by the action of monodromy.
Let
\begin{equation} \label{Geq} \mathrm{Aut}(H_{\ZZ},Q) := \{g\colon H_{\Z} \to H_{\Z} \mid Q(g\xi, g\eta) = Q(\xi, \eta)\ \mathrm{for}\ \mathrm{all}\ \xi,\eta \in  H_{\Z}\}.\end{equation}
This group acts on $\calD$. The map $f\colon \calX \to B$ defines a fibre bundle on $B$, or a locally constant sheaf of abelian groups
\[\calH_{\Z}^n = \bigcup_{b \in B} H^n(X_b,\Z) / (\text{torsion}) = R^nf_*\Z_{\calX} / (\text{torsion}) \] 
which in turn defines an action of the fundamental group $\pi_1(B)$ on a fibre $H_{\Z} = H^n(X_b,\Z) / (\text{torsion})$.
We have a monodromy representation
\[\varrho \colon \pi_1(B) \to \mathrm{Aut}(H_{\ZZ},Q).\]

Suppose that $\Gamma \subset \mathrm{Aut}(H_{\ZZ},Q)$ is the smallest subgroup containing the image of $\varrho$. We call it the \emph{Monodromy group}. Then there is a well-defined map $\phi\colon B \to \Gamma \backslash \calD$ given by $b \mapsto [F^\bullet_b]$, where $F^\bullet_b$ is the hodge filtration of $P^k(X_b, \CC)$.
This is called the \emph{global period mapping}. The quotient $ \Gamma \backslash \calD$ is called the \emph{global period domain} for the family $f$.

We now calculate global period domains in our examples.

\begin{es} We first look at the case of an elliptic curve $E$, the group $\mathrm{Aut}(H_{\ZZ},Q)$ is the group of linear transformations $\Z^2 \to \Z^2$ that preserve the bilinear form $Q$. This is precisely the group $\mathrm{SL}(2,\Z)$ since for a linear transformations $\varphi \in \mathrm{Aut}(H_{\ZZ},Q) $, $\varphi$ has to be rapresented by a matrix $\left( \begin{array}{cc} a & b \\ c & d \end{array} \right) \in M_{2\times 2}(\Z)$ which satisfies $Q(\varphi(x),\varphi(y))=Q(x,y) \quad \forall x,y \in \ZZ^2$, i.e. 
\[
\left( \begin{array}{cc} a & b \\ c & d \end{array} \right)^T  \cdot \left( \begin{array}{cc} 0 & 1 \\ -1 & 0 \end{array} \right) \cdot \left( \begin{array}{cc} a & b \\ c & d \end{array} \right) = \left( \begin{array}{cc} 0 & 1 \\ -1 & 0 \end{array} \right)
\]
which happens if and only if $ad-bc=1$, or $\varphi \in  \mathrm{SL}(2,\Z)$.

Recall that we identified a point $F^1$ in $\calD$ with an element $\lambda \in H_\CC$ that spans $F^1$ and we saw that the in the standard basis $ \{ \alpha, \beta \}$ we can write $\lambda$ as $\alpha + z\beta$ with $\im z > 0$. 
We have that $\lambda=z_1\alpha + z_2\beta$,  $\varphi \cdot \lambda = (az_1+bz_2)\alpha + (cz_1+dz_2) \beta $. Hence the general element $\varphi$ sends  $\alpha + z\beta$ to $(a+bz)\alpha + (c+dz) \beta$ which is linearly equivalent to $\alpha + \frac{c+dz}{a+bz} \beta$.

In other words this group acts on $\calD \cong \mathfrak{H}$ by
\[ \left( \begin{array}{cc} a & b \\ c & d \end{array} \right) \cdot z = \frac{c + dz}{a + bz}. \]

Note that the negative identity matrix acts trivially, so we get an induced action of the modular group $\Gamma := \mathrm{PSL}(2,\Z)$ on $\calD$. For a family of elliptic curves, the image of the monodromy will be contained in $\mathrm{PSL}(2,\Z)$, so we can say that the period domain for a family of elliptic curves will be contained in 
\[ \Gamma \backslash \calD \cong \mathrm{PSL}(2,\Z) \backslash \mathfrak{H}.\]
\end{es}

\begin{es}[Legendre family]
Consider the Legendre family of elliptic curves, given by the polynomial
\[
y^2=x(x-1)(x-b).
\]
We homogenize it and set
 \[
X \defeq  \left\{ \quad ([x_0, x_1, x_2], b) \in \PP^2 \times B \quad | \quad x_2^2x_0 = x_1(x_1 - x_0)(x_1 - bx_0) \quad \right\},
 \]
where $B = \C \setminus \{0, 1\}$ with $\pi \colon X \to B$ the projection. Then $X \xrightarrow{\pi} B$ is a family and the fibre $X_b$ is a smooth plane cubic with $J$-invariant
\[
J(X_b) = \frac{(b^2 - b + 1)^3}{b^2(b - 1)^2} .
\]
Note that the fibres are diffeomorphic, but in general not biholomorphic since we know that two elliplitic curves are biholomorhic if and only if they have the same $J$-invariant.

The monodromy group $\Pi$ is the smallest subgroup containing the image of the map
\[
\rho \colon \pi_1(\CC \setminus \{0,1\},p) \lra \mathrm{SL}(2,\Z).
\]
Here $\pi_1(\CC \setminus \{0,1\},p)$ is a free abelian goup on two generators representing the loops around $0$ and $1$,. Let's denote them $a$ and $b$ . One can calculate, using the Picard-Lefschetz formula, see \cite[Sect. 1]{CMSP03}, that their images are
\begin{align*}
    A=\rho(a)= \left( \begin{array}{cc} 1 & 2 \\ 0 & 1 \end{array} \right) \qquad \text{ and } \qquad B=\rho(b)= \left( \begin{array}{cc} 1 & 0 \\ -2 & 1 \end{array} \right).
\end{align*}
Therefore the monodromy group $\Pi$ is $\langle A,B \rangle$.
The two matrices are congruent modulo $2$ to the identity matrix, so every matrix in $\Pi$ has this property. Let
\[
\Gamma(N) \defeq  \{ M \in \mathrm{SL}(2, \Z) | M \equiv Id \mod N \}.
\]
It is a normal subgroup of finite index in $\mathrm{SL}(2, \Z)$ and  we have seen that $\Pi$ is a subgroup of $\Gamma(2)$.
One can actually see (cfr. \cite{Cl80}) that $\Pi = \Gamma(2)$, and that it is a subgroup of index $6$ in $\mathrm{SL}(2, \Z)$, hence of index $3$ in $\mathrm{PSL}(2, \Z)$.
\end{es}

\begin{es} Now let's look at the case of a curve $C$ of genus $g \geq 1$. The group $\mathrm{Aut}(H_{\ZZ},Q)$ is the group of linear transformations $\Z^{2g} \to \Z^{2g}$ that preserve the bilinear form $Q$, which is precisely the group $\mathrm{Sp}(2g,\Z)$
 since for a linear transformations $\varphi \in \mathrm{Aut}(H_{\ZZ},Q) $, $\varphi$ has to be rapresented by a matrix $\left( \begin{array}{cc} A & B \\ C & D \end{array} \right) \in M_{2g\times 2g}(\Z)$ which satisfies $Q(\varphi(x),\varphi(y))=Q(x,y) \quad \forall x,y \in \ZZ^{2g}$, i.e. 
\[
\left( \begin{array}{cc} A & B \\ C & D \end{array} \right)^T  \cdot \left( \begin{array}{cc} 0 & I_g \\ -I_g & 0 \end{array} \right) \cdot \left( \begin{array}{cc} A & B \\ C & D \end{array} \right) = \left( \begin{array}{cc} 0 & I_g \\ -I_g & 0 \end{array} \right)
\]
which is the condition for $\varphi$ to be in  $\mathrm{Sp}(2g,\Z)$.


Recall that we identified a point $F^1$ in $\calD$ with a basis $ \{ \lambda_1, \dots , \lambda_g \} \subset H_\CC$ for $F^1$ and we saw that the in the standard basis $ \{ \alpha_i, \beta_j \}$ we can write $(\lambda_1, \dots , \lambda_g)$ as $(\alpha_1, \dots , \alpha_g ,\beta_1, \dots , \beta_g) \left(\begin{array}{c} I_g \\ Z \end{array} \right)$ with $Z$ symmetric and $\mathfrak{Im} Z$ positive definite. 

We have that for $(\lambda_1, \dots, \lambda_g)=(\alpha_1, \dots , \alpha_g ,\beta_1, \dots , \beta_g) \left(\begin{array}{c} Z_1 \\ Z_2 \end{array} \right)$, $\varphi$ acts as $\varphi \cdot \lambda = (\alpha_1, \dots , \alpha_g ,\beta_1, \dots , \beta_g) \left(\begin{array}{c} A Z_1 + B Z_2 \\ CZ_1 + D Z_2 \end{array} \right) $. Hence the general element $\varphi$ sends  $ \left(\begin{array}{c} I_g \\ Z \end{array} \right)$ to $(\alpha_1, \dots , \alpha_g ,\beta_1, \dots , \beta_g) \left(\begin{array}{c} A + B Z \\ C + D Z \end{array} \right)$ which is linearly equivalent to 
\[ (\alpha_1, \dots , \alpha_g ,\beta_1, \dots , \beta_g) \left(\begin{array}{c} I_g \\ {(C + D Z)}{(A +BZ)^{-1}}  \end{array} \right).
\]

In other words this group acts on $\calD \cong \mathfrak{H}_g$ by
\[ \left( \begin{array}{cc} A & B \\ C & D \end{array} \right) \cdot Z = (C + DZ)(A + BZ)^{-1}, \]
Note that the negative identity matrix $-I_{2g}$ acts trivially, so we get an induced action of the group $\Gamma_g := \mathrm{Sp}(2g,\Z) / \{\pm I_{2g}\}$ on $\calD$. The period domain for a family of curves of genus $g \geq 1$ is contained in
\[ \Gamma \backslash \calD \cong \Gamma_g \backslash \mathfrak{H}_g.\]

This is not a complex manifold because of the presence of fixed points of the action. However, since $\mathrm{Sp}(2g, \Z) \subset \mathrm{Sp}(2g, \RR)$ is discrete and $U(g)$ is compact it can be shown (\cite[Sect. 1.2]{CMSP03}), that the action is properly discontinuous, i.e. we obtain an analytic space.
\end{es}

Define $\calM_g= \left\{ \text{isomorphism classes of genus g riemann surfaces} \right\}$ and $\calA_g \defeq \mathrm{Sp}(2g, \Z) \backslash \calH_g = \mathrm{Sp}(2g, \Z) \backslash  \mathrm{Sp}(2g, \RR) /U(g)$.
\begin{defn}
    The Torelli map 
    \[
    \calM_g \lra \calA_g
    \]
    is defined by sending $[C]$ to the polarized Hodge structure on $H^1(C, \Z)$. 
\end{defn}
Choosing a symplectic basis $\{\alpha_i,\beta_j\}$ of $H^1(C, \Z)$ means giving an isometry of polarized lattices $\phi \colon (H^1(C, \Z), Q) \to (\Z^{2g}, \cdot)$ to the standard symplectic lattice. Then to a class $[C]$ we associate the point of $\mathfrak{H}_g$ given by $\phi(H^{1,0})$. Since $\phi$ is unique up to post-compostion by elements of $\mathrm{Sp}(\Z^{2g}, \cdot)$, the corresponding point $\phi(H^{1,0}) \in \mathfrak{H}_g$ is well-defined up to  $\mathrm{Sp}(\Z^{2g}, \cdot)$, i.e. its class in $\calA_g$ is well-defined.

It’s a fact from complex geometry that deformations of $C$ as a Riemann surface are identified with $H^1(C, T_C)$. Serre duality gives $h^1(C, T_C) = h^0(C, K^{\otimes2}_C)$ and Riemann-Roch gives
\[
h^0(K^{\otimes2}_C) - h^1(K^{\otimes2}_C) = \deg (K^{\otimes2}_C) + 1 - g = 3g - 3.
\]
Again by Serre duality, $h^1(K^{\otimes2}_C) = h^0(K^{-1}_C) = 0$ if $g \geq 2$, so $h^0(K^{\otimes2}_C) = 3g - 3$ for $g \geq 2$. Thus $\dim \calM_g = 3g - 3$, whereas $\dim \calA_g = \dim \mathfrak{H}_g = \dim \Sym_{g\times g}(\CC) = {g+1\choose 2}$.

We can list the values for small values of $g$:

\begin{center}
    
\begin{tabular}{c|c|c|c|c|c|c}
        $g$       &  0   &  1   &  2   &  3  &  4  &  5  \\ \hline
   $\dim \calM_g$ &  0   &  1   &  3   &  6  &  9  &  12 \\ \hline
   $\dim \calA_g$ &  0   &  1   &  3   &  6  &  10 &  15 \\
\end{tabular}
\end{center}

\begin{theorem}[Torelli Theorem] \textup{\cite[Thm. 3.5.2]{CMSP03}} \label{Torelli}


    The Torelli map is injective.
\end{theorem}

\subsection{Abstract Variations of Hodge Structure} \label{subsecvhs}

We now want to generalize what happens in the case of \emph{geometric} variations of Hodge structure, i.e. the ones coming from a family of compact K\"{a}hler manifolds. 
We therefore come to the abstract definition of variations of polarized Hodge structure as given in \cite{PS08}[Ch. 10].

Let $B$ be a complex manifold and let $\calE_{\Z}$ be a local system, i.e. a locally constant sheaf of finitely generated free $\Z$-modules on $B$. Define $\calE := \calE_{\C} \otimes \calO_B$ where $\calE_{\C} := \calE_{\Z} \otimes \C$ . Then $\calE$ is a holomorphic vector bundle which carries the natural flat connection $\nabla\colon \calE \to \calE \otimes \Omega_B^1$ induced by $d=\partial \colon \calO_B \to \Omega_B^1$. 
We have a correspondence (equivalence of categories) between complex local systems on $B$ and holomorphic vector bundles on $B$ equipped with a flat connection. To a local system $\calE_{\Z}$ we associate $(\calE, \nabla)$. To a holomorphic vector bundle with a flat connection $(\calV,\nabla)$ we associate $\VV \defeq \calV^\nabla= \ker \nabla $, the local system of horizontal sections. 

Let $\{\calF^p\}$ be a finite decreasing filtration of $\calE$ by holomorphic sub-bundles.

\begin{defn} \label{defn:VHS} The data $(\calE_\ZZ,\calF)$ of a local system on $B$ together with a finite decreasing filtration of  the holomorphic vector bundle $\calE := \calE_{\C} \otimes \calO_B$  defines a \emph{variation of Hodge structure of weight $n$} on $B$ if
\begin{enumerate}[(i)]
\item $\{\calF^p\}$ induces Hodge structures of weight $n$ on the fibres of $\calE$, and
\item  the connection $\nabla \colon \calE \to \calE \otimes_{\calO_B} \Omega^1_B$ whose sheaf of horizontal sections is $\calE_\CC$ satisfies \textbf{Griffiths transversality}:
\[ \nabla(\calF^p) \subset \calF^{p-1} \otimes \Omega^1_B
\]
i.e. if $s$ is a section of $\calF^p$ and $\zeta$ is a vector field of type $(1,0)$, then $\nabla_{\zeta}s$ is a section of $\calF^{p-1}$.
\end{enumerate}
\end{defn}
In the case of the geometric setting $f \colon \calX \to B$ we take $\calE_\Z$ as  $\calH_{\Z}^n = \bigcup\limits_{b \in B} H^n(X_b,\Z) = R^nf_*\Z_{\calX}$, where $\Z$ is the constant sheaf of stalk $\Z$, and $R^nf_*$ is the $n$-th derived functor of the functor $f_*$ from the category of sheaves over $\calX$ to the category of sheaves over $B$.

\begin{proposition} \textup{\cite{PS08}[Prop. 10.24]}
    Let $f \colon \calX \to B$ be as above. There is a unique flat connection, the Gauss-Manin connection $\nabla\colon \calH^n \to \calH^n \otimes \Omega_B^1$ whose sheaf of locally constant  (or horizontal) sections is $\calH^n_\C = R^n f_* \C_X$. This is a locally constant sheaf whose fibre at $b \in B$ is $H^n(X_b, \C)$.
\end{proposition}
\begin{proof}
We define the Gauss-Manin connection as the natural connection on $R^n f_* f^* \calO_B \simeq R^n f_* \CC_\calX \otimes_{\CC_B} \calO_B = \calH^n$ coming from $d \colon f^*\calO_B \to f^*\Omega^1_B$ where $f^*$ is the inverse image functor of sheaves and the isomorphisms can be checked at the level of stalks.
Pulling back the exact sequence of sheaves
\[
0 \to \C_B \to \calO_B \xrightarrow{d} \Omega^1_B\xrightarrow{d} \Omega^2_B,
\]
we get the exact sequence
\[
0 \to f^*\C_B \to f^*\calO_B \xrightarrow{d} f^*\Omega^1_B\xrightarrow{d} f^*\Omega^2_B.
\]
Taking its $n$-th direct image one finds on $B$ the exact sequence
\[
0 \to R^n f_*f^* \C_B \to R^n f_*f^*\calO_B \xrightarrow{\nabla} R^n f_*f^*\Omega^1_B\xrightarrow{\nabla} R^n f_*f^*\Omega^2_B
\]
Hence $\nabla \circ \nabla=0$, i.e. $\nabla$ is a flat
connection. Since its locally constant sections generate the sheaf $R^n f_*f^* \C_B$, this identifies $\nabla$ with the unique flat connection on $R^n f_*f^*\calO_B$ whose sheaf
of locally constant sections is $R^n f_*f^* \C_B =  \calH^n_\C$.
\end{proof}

We next introduce the relative De Rham complex.
\begin{defn}
    Let $f \colon X \to Y$ be a holomorphic map between complex
manifolds. The relative de Rham complex of $f$, denoted by $\Omega^\bullet_{X/Y}$, is the quotient of the K\"ahler de Rham complex $\Omega^\bullet_X$ by the subcomplex generated locally by forms $f^*(\eta) \wedge \omega$ where $\eta$ is a local section of $\Omega^1$ and $\omega$ a local section of $\Omega^{\bullet-1}_X$:
\[
\Omega^p_{X/Y} = \frac{ \Omega^p_X}{f^*\Omega^1_Y \wedge \Omega^{p-1}_X}
\]
If $X$ and $Y$ are smooth and $f$ is of maximal rank, then in suitable local coordinates on $X$ and $Y$ the map $f$
has the form $(z_1, \dots ,z_n) \mapsto (z_1, \dots ,z_k)$ with $n = \dim(X)$, $k = dim(Y)$ so that $\Omega^\bullet_{X/Y}$ is locally isomorphic to the exterior algebra over $\calO_X$ on the generators $dz_{k+1},\dots ,dz_n$.
\end{defn} 

We have an alternative description of $\calH^n$ given by a proposition which we state without proof:
\begin{proposition} \textup{\cite[Prop 2.28]{Del70}}
    Let $f \colon X \to B$ be a smooth proper holomorphic map between complex manifolds, $n \in \NN$ and $\calE_\C$ a local system of complex vector spaces on $X$. There is a natural isomorphism
    \[
    \calO_B \otimes_\CC R^n f_* \calE_\CC \simeq R^n f_* ( \Omega^\bullet_{X/B } \otimes_\CC \calE_\CC )
    \]
\end{proposition}
The isomorphism is the composition 
\[
\calO_B \otimes_\CC R^n f_* \calE_\CC \to  R^n f_* f^*\calO_B \otimes_\CC R^n f_* \calE_\CC   \cong R^n f_* ( f^*\calO_B \otimes_\CC \calE_\CC )
\]
of the natural map coming from the adjunction morphism $\calO_B \to  R^n f_* f^*\calO_B $ composed with the isomorphism 
\[
R^n f_* ( f^*\calO_B \otimes_\CC \calE_\CC ) \xrightarrow{\sim} R^n f_* ( \Omega^\bullet_{X/B } \otimes_\CC \calE_\CC ) 
\]
which is an isomorphism since $\Omega^\bullet_{X/B }$ is a resolution of $ f^*\calO_B$.

As a corollary of the theorem we have:
\[
\calH^n \simeq R^n f_*(\Omega^\bullet_{X/B}).
\]
If $\dim B = 1$ we have an exact sequence
\begin{align}\label{SESRelDeRham}
0 \to f^*(\Omega^1_B) \otimes \Omega^\bullet_{X/B}[-1] \xrightarrow{\wedge} \Omega^\bullet_X \to \Omega^\bullet_{X/B} \to 0.
\end{align}

\begin{theorem}  \label{GaussManinThm}
     The Gauss-Manin connection $\nabla$ is the connecting homomorphism

     \begin{equation} \label{GaussManin}
\xymatrix{
   R^n f_*\Omega^\bullet_{X/B} \ar@{=}[d] \ar[r]^-{\delta} & R^{n+1} f_* (f^*(\Omega^1_B) \otimes_{\calO_X} \Omega^\bullet_{X/B}[-1]) \ar@{=}[d]    \\ 
   R^n f_*\Omega^\bullet_{X/B} \ar[r]_-{\nabla} & \Omega^1_B \otimes_{\calO_B} R^n f_* \Omega^\bullet_{X/B} 
}
\end{equation}
in the long exact sequence obtained by applying $Rf_*$ to the exact sequence (\ref{SESRelDeRham}).
\end{theorem}
\begin{proof}
    See \cite[Thm. 2]{KO68}.
\end{proof}
The isomorphism on the right comes from the fact that  $\Omega^1_B$ is locally free and the differential in the complex $f^*(\Omega^1_B) \otimes_{\calO_X} \Omega^\bullet_{X/B}[-1]$ is $f^*\calO_B$-linear.

We now define the Hodge filtration on $R^n f_*\Omega^\bullet_{X/B}$; it is obtained as follows. The relative de Rham complex inherits the trivial filtration $\sigma$ from the absolute de Rham complex.
\[
\sigma^{\geq p} \Omega^\bullet_{X/B} \defeq \{ 0 \to \cdots \to 0 \to \Omega^p_{X/B} \to \Omega^{p+1}_{X/B} \to \cdots  \}
\]

We define
\begin{equation}
\calF^p \calH^n = \calF^p R^n f_* \Omega^\bullet_{X/B} := im (R^n f_* \sigma^{\geq p} \Omega^\bullet_{X/B} \to R^n f_* \Omega^\bullet_{X/B}) \label{HdgFiltr}
\end{equation}

We now can prove :
\begin{cor}[Griffiths’ transversality theorem] \textup{\cite[Cor.10.31]{PS08}}
     The Gauss-Manin connection has the property
     \[
     \nabla(\calF^p\calH^n) \subseteq \Omega^1_B \otimes \calF^{p-1}\calH^n.
     \]
\end{cor}
\begin{proof}
    This follows from the filtered version of (\ref{SESRelDeRham})
    \[
0 \to f^*\Omega^1_B \otimes \sigma^{\geq p-1}\Omega^\bullet_{X/B}[-1] \to \sigma^{\geq p} \Omega^\bullet_X \to \sigma^{\geq p}\Omega^\bullet_{X/B} \to 0
    \]
after applying $R f_*$:
\begin{equation*}
\xymatrix{
   \calF^p R^n f_*(\Omega^\bullet_{X/B}) \ar@{=}[d]  \ar[r]^-{\delta} & \calF^{p-1} R^n f_*(f^*\Omega^1_B \otimes \Omega^\bullet_{X/B}) \ar@{=}[d]  \\ 
   \calF^p \calH^n  \ar[r]^-{\nabla} &  \Omega^1_B \otimes \calF^{p-1} \calH^n
}
\end{equation*}
Indeed $\delta$ goes from $R^n f_*( \sigma^{\geq p}\Omega^\bullet_{X/B})$ to $R^{n+1} f_* ( f^*\Omega^1_B \otimes \sigma^{\geq p-1}\Omega^\bullet_{X/B}[-1]) \simeq \Omega^1_B \otimes  R^n f_* ( \sigma^{\geq p-1}\Omega^\bullet_{X/B}) $. Hence taking the images in $R^n f_*( \Omega^\bullet_{X/B})$, we see that $\delta$, i.e. the connection $\nabla$, takes $\calF^p\calH^n$ to $\Omega^1_B \otimes  R^n f_* ( \sigma^{\geq p-1}\Omega^\bullet_{X/B}) = \Omega^1_B \otimes \calF^{p-1}\calH^n $.
\end{proof}
Hence we found:
\begin{cor}
     Let $f \colon X \to B$ be a proper and smooth holomorphic map between complex manifolds, let $n \in \NN$. Suppose that X a K\"ahler manifold. Then, with the Hodge filtration (\ref{HdgFiltr}), the local system $\calH^n_\Z = R^n f_* \Z_X$ underlies a variation of Hodge structure.
\end{cor}

 



\ifSubfilesClassLoaded{
}{} % la seconda {} è per else

\end{document}
